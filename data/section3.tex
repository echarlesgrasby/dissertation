%%%%%%%%%%%%%%%%%%%%%%%%%%%%%%%%%%%%%%%%%%%%%%%%%%%
%
%  New template code for TAMU Theses and Dissertations starting Spring 2021.  
%
%
%  Author: Thesis Office
%  
%  Last Updated: 1/13/2021
%
%%%%%%%%%%%%%%%%%%%%%%%%%%%%%%%%%%%%%%%%%%%%%%%%%%%
%%%%%%%%%%%%%%%%%%%%%%%%%%%%%%%%%%%%%%%%%%%%%%%%%%%%%%%%%%%%%%%%%%%%%%
%%                           SECTION III
%%%%%%%%%%%%%%%%%%%%%%%%%%%%%%%%%%%%%%%%%%%%%%%%%%%%%%%%%%%%%%%%%%%%%

\chapter{\MakeUppercase{Requirements Modeling and Analysis}}
\label{cha:requirements}

This chapter contains the information learned from conducting the Delphi study component of this research. The purpose of the Delphi study was to gather expert opinions from the professionals that work in Market Monitoring; it is a niche industry.

The study itself aligns with the E-Delphi variant of the Delphi methodology. As such, the expert panel was distributed across the globe. The target population for recruitment came from the Energy Intermarket Surveillance Group (EISG), a professional affiliation organization of individuals that work in energy market monitoring.

Prospective participants were contacted via the EISG forum website, through both a forum posting (inviting members to sign-up for the survey study) and an email distribution list. Volunteers were instructed to contact the researcher to "opt-in" for participation. They were furnished with a Waiver of Informed Consent, which outlined both the purpose and the benefits and risks of this study. Once confirmed for participation, responders were provided with a link to Google Forms which managed the survey questions and collected results for the researcher to process.

Lessons learned: 

As it happened, it took longer to seat the panel than it took to actually gather and analyze the responses. For others that wish to utilize the Delphi Methodology, finding a way to speed up the seating of the panel should aid in a quicker path to completion.