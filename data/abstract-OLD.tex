%%%%%%%%%%%%%%%%%%%%%%%%%%%%%%%%%%%%%%%%%%%%%%%%%%%
%
%  New template code for TAMU Theses and Dissertations starting Spring 2021.  
%
%
%  Author: Thesis Office
%  
%  Last Updated: 1/13/2021
%
%%%%%%%%%%%%%%%%%%%%%%%%%%%%%%%%%%%%%%%%%%%%%%%%%%%
%%%%%%%%%%%%%%%%%%%%%%%%%%%%%%%%%%%%%%%%%%%%%%%%%%%%%%%%%%%%%%%%%%%%%
%%                           ABSTRACT 
%%%%%%%%%%%%%%%%%%%%%%%%%%%%%%%%%%%%%%%%%%%%%%%%%%%%%%%%%%%%%%%%%%%%%

\chapter*{ABSTRACT}
\addcontentsline{toc}{chapter}{ABSTRACT} % Needs to be set to part, so the TOC doesn't add 'CHAPTER ' prefix in the TOC.

\pagestyle{plain} % No headers, just page numbers
\pagenumbering{roman} % Roman numerals
\setcounter{page}{2}

\indent This dissertation aims to investigate an intersection between several domains of expertise: 1) data and information quality, 2) language engineering, 3) electricity market monitoring, 4) knowledge management, and 5) design science research. Electricity markets represent unique research opportunities in data and information quality. Not only do electricity markets rely on data-driven decisions to perform operations, but the markets themselves are paramount to the well-being of critical infrastructure in the United States. Additionally, electricity markets facilitate transactions in excess of hundreds of millions of dollars each year.

Due to the lucrative nature of these marketplaces, there are abundant opportunities for both market manipulation and gaming. Market monitoring units (MMUs) perform routine market surveillance to identify instances of fraud, market manipulation and gaming, and refer these bad actors to regulators. This research investigates the efficacy of both designing and implementing a domain-specific language to perform electricity market analytics for the purposes of market monitoring. The purpose of this system is to abstract complex, data-oriented tasks away from a market analyst or economist.



 

\pagebreak{}
