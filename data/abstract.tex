%%%%%%%%%%%%%%%%%%%%%%%%%%%%%%%%%%%%%%%%%%%%%%%%%%%
%
%  New template code for TAMU Theses and Dissertations starting Spring 2021.  
%
%
%  Author: Thesis Office
%  
%  Last Updated: 1/13/2021
%
%%%%%%%%%%%%%%%%%%%%%%%%%%%%%%%%%%%%%%%%%%%%%%%%%%%
%%%%%%%%%%%%%%%%%%%%%%%%%%%%%%%%%%%%%%%%%%%%%%%%%%%%%%%%%%%%%%%%%%%%%
%%                           ABSTRACT 
%%%%%%%%%%%%%%%%%%%%%%%%%%%%%%%%%%%%%%%%%%%%%%%%%%%%%%%%%%%%%%%%%%%%%

\chapter*{ABSTRACT}
\addcontentsline{toc}{chapter}{ABSTRACT} % Needs to be set to part, so the TOC doesn't add 'CHAPTER ' prefix in the TOC.

\pagestyle{plain} % No headers, just page numbers
\pagenumbering{roman} % Roman numerals
\setcounter{page}{2}

\indent This dissertation aims to investigate an intersection between several domains of expertise as they related to electric markets and electric market monitoring. These domains include: 1) data and information quality, 2) language engineering, 3) knowledge management, and 4) design science research. Electricity markets present unique research opportunities for deriving new value from data and information assets, due to the enormous volume of data generated, each day, by public utilities, transmission and market operators, and regulators. 

These data assets are critically necessary to both ensure reliable operation of the North American Bulk Electric System (BES), as well as operate efficient markets for selling energy on a wholesale scale. The outcomes of these markets have tangible, long-term impacts on system reliability, prices set for residential and commercial energy consumers, and electricity infrastructure development. 

Due to the lucrative nature of energy marketplaces, there are abundant opportunities for market participants to manipulate these markets. Market monitors are the group of professionals that work to ensure that the BES remains a fair, efficient, and open-access marketplace in the face of such manipulative action. They perform routine surveillance and forensic analysis to identify instances of fraud and market manipulation and refer such cases to regulatory authorities. 

This research investigates the efficacy of designing and implementing a layer of abstraction between market monitors and the data that they use to perform their forensic analysis. Using domain-specific language engineering and information quality principles, the investigator expects to develop a system to ease data manipulation for market monitoring analysts and economists.

\pagebreak{}
