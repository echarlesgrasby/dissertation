%%%%%%%%%%%%%%%%%%%%%%%%%%%%%%%%%%%%%%%%%%%%%%%%%%%
%
%  New template code for TAMU Theses and Dissertations starting Spring 2021.  
%
%
%  Author: Thesis Office
%  
%  Last Updated: 1/13/2021
%
%%%%%%%%%%%%%%%%%%%%%%%%%%%%%%%%%%%%%%%%%%%%%%%%%%%
%%%%%%%%%%%%%%%%%%%%%%%%%%%%%%%%%%%%%%%%%%%%%%%%%%%%%%%%%%%%%%%%%%%%%
%%                           ABSTRACT 
%%%%%%%%%%%%%%%%%%%%%%%%%%%%%%%%%%%%%%%%%%%%%%%%%%%%%%%%%%%%%%%%%%%%%

\chapter*{ABSTRACT}
\addcontentsline{toc}{chapter}{ABSTRACT} % Needs to be set to part, so the TOC doesn't add 'CHAPTER ' prefix in the TOC.

\pagestyle{plain} % No headers, just page numbers
\pagenumbering{roman} % Roman numerals
\setcounter{page}{2}

\indent This dissertation investigates an intersection between multiple domains of expertise related to electric markets and electric market monitoring. These domains are: 1) data and information quality, 2) language engineering, and 3) knowledge management. Electricity markets present unique opportunities for deriving new value from data and information assets, due to the enormous volume of data generated, each day, by public utilities, transmission and market operators, and regulators. These data assets are critically necessary to both ensuring reliable operation of the North American Bulk Electric System (BES), as well as operating efficient markets for selling energy on a wholesale scale. The outcomes of these markets have tangible, long-term impacts on system reliability, prices for residential and commercial energy consumers, and electricity infrastructure development. 
%%TODO: 2025-06-27 - reword "that the BES remains..." to refer to the market on top of the BES instead. This is in the Introduction chapter too.
Due to the complex nature of energy marketplaces, there are abundant opportunities for market participants to manipulate these markets. Market monitors are the professionals who work to ensure that the markets utilizing the BES remain fair, efficient, and open-access in the face of such manipulative actions. They perform routine surveillance and forensic analysis to identify instances of fraud and market manipulation and refer such cases to regulatory authorities. 

This research investigates the efficacy of designing and implementing a layer of abstraction between market monitors and the data that they use to perform their forensic analysis. Using domain-specific programming language engineering and information quality principles, the intent is to develop a system to ease data manipulation for market monitors.

\pagebreak{}
